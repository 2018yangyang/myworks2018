\jxhj{%教学后记
	}
\skrq{%授课日期
	2017年11月9日 4-5节}
\ktmq{%课题名称
	 型腔加工}
\jxmb{%教学目标,每行前面要加 \item
	\item 掌握挖槽加工的工艺;
	\item 掌握挖槽加工的下刀方式;
	\item 会写简单的挖槽程序。}
\jxzd{%教学重点,每行前面要加 \item
	\item 挖槽加工的工艺;
	\item 写简单的挖槽程序。 }
\jxnd{%教学难点,每行前面要加 \item
	\item 写简单的挖槽程序。 }
\jjff{%教学方法
	通过讲述、举例、演示法来说明;}

\makeshouye %制作教案首页

%%%%教学内容
\subsection{组织教学}
\begin{enumerate}[\hspace{2em}1、]
	\item 集中学生注意力;
	\item 清查学生人数;
	\item 维持课堂纪律;
\end{enumerate}

\subsection{复习导入及主要内容}
\begin{enumerate}[1、]
\item 增加刀路去残料;
\item 加工实例;
\item 相关计算。
\end{enumerate}

\subsection{教学内容及过程}
\subsubsection{刀具的选择}
要求:有底刃和侧刃(即键槽铣刀)两条侧刃

但粗加工还是可以用立铣刀,(要先加工工艺孔)

\subsubsection{下刀方式}
1、直接下刀:一轴移动(预钻孔) 

G01 Z\_\_ F(10-20);

2、斜线下刀:两轴移动(狭长地带)

G01 Z\_\_ X\_\_ / Y\_\_ F(30-40)

3、螺线下刀:三轴移动(空间较大)

G17 G02/G03 X\_\_ Y\_\_ Z\_\_ I\_\_ J\_\_ / R\_\_ F\_\_

指令要写全,不能省。

\subsubsection{斜线下刀}

1、坡度3°\~ 5°

即:移动10mm下刀约1mm

2、长度不够:

可采用Z字形下刀

3、斜坡铣削:

G01 X\_\_ Z\_\_ F\_\_

X\_\_  (返回)

\subsubsection{去材料方式}

1、往复铣削(如铣长方形)

2、环形(如铣圆形,正方形)

3、少用刀补来去材料

\subsubsection{圆形槽铣削}

利用数控铣床加在\diameter 110*35的毛坯上加工\diameter 100*5的槽。毛坯材料为45号钢,上表面未加工。按图样要求完成零件的节点、基点、辅助点计算。设定工件坐标系,制定正确的工艺方案(包括定位,夹紧方案和工艺路线),选择合理的刀具和切削工艺参数,编写数控加工程序。

1、工艺分析:

零件简单,尺寸精度均达到IT8-IT7级

采用机用平口钳装夹,两侧面先工艺平面用于装夹

工件坐标系原点设在铣好的上表面中心

工序:平面

挖槽:

2、刀具选择:

平面用\diameter 80可转位铣刀,

挖槽:粗加工用\diameter 14三刃立铣刀

精加工用\diameter 12四刃立铣刀

3、刀削参数的选择:

{\noindent
    
     \begin{figure}[!hbtp]
        \centering
        \footnotesize
        %\hspace{-3.4ex}
        \renewcommand\arraystretch{1.9}
\begin{tabu}to 0.6\textwidth{|c|c|c|c|c|c|c|c|}
    \hline 
    序号 & 加工 & 刀具类型 & 刀具 & 主轴 & 进给& 长度 & 半径 \\ 
    & 内容 &  & 材料 & 转速 & 速度 & 补偿 & 补偿 \\
    \hline 
    1 & 粗加工上表面 & \diameter 可转位铣刀 & 硬质合金 & 500 & 250 &  &  \\ 
    \hline 
    2&精加工上表面& \diameter 可转位铣刀  & 硬质合金 &800  &160  &  &  \\ 
    \hline 
    3&粗加工槽  & \diameter 14立铣刀 &高速钢  &500  &80  &  &7.2  \\ 
    \hline 
    4&精加工槽  & \diameter 12立铣刀 &高速钢  &800  &80  &  &5.985  \\ 
    \hline 
\end{tabu} 
\end{figure}}

4、走刀路径

粗加工刀补:7.2  R0=42.8

去残料圆半径   R1=35

R2=27

R3=19

R4=11

5、参考程序

\begin{lstlisting}
O0001;(粗加工)
G54 G17 G40 G49 G90;
M03 S500;
G01 Z30.0 F100;
X-42.8 Y0;
Z0.5;
G01 X42.8 Z-5.0 F40;
X-42.8 F80;
X-11.0;
G02 I11.0;
G01 X-19.0;
G02 I19.0;
G01 X-27.0;
G02 I-27.0;
G01 X-35;
G02 I35.0;
G01 X-42.8; 
G02 I42.8;
G0 Z30.0;
M05;
M30;
\end{lstlisting}

\begin{lstlisting}
O0002;(精加工)
G54 G17 G40 G49 G90
M03 S800
G01 Z30.0 F100
XO Y0
Z-5.0
G01 G41 X40.0 Y-10.0 D1 
G03 X50.0 Y0 R10.0
G03 I50.0
G03 X40.0 Y10.0 R10.0
G40 G01 X0 Y0
Z30.0
M05
M30
\end{lstlisting}

\subsection{课堂小结}
\begin{enumerate}[1、]
	\item 刀具的选择;
	\item 下刀方式;
	\item 斜线下刀;
    \item 去残料方式;
    \item 圆形槽铣削加工实例。
\end{enumerate}

\vfill
\subsection{布置作业}
\begin{enumerate}[1、]
	\item 编写圆形槽的程序。
\end{enumerate}
\vfill