%%%%%%%%------------------------------------------------------------------------
%%%% 日常所用宏包

%% 控制页边距
\usepackage[top=2cm, bottom=2cm, left=2.cm, right=2.cm,includehead,includefoot]{geometry}

\usepackage{listing}

%% 控制项目列表
\usepackage{enumerate}

%% 多栏显示
\usepackage{multicol}

%% hyperref宏包,生成可定位点击的超链接,并且会生成pdf书签
\usepackage[%
    pdfstartview=FitH,%
    CJKbookmarks=true,%
    bookmarks=true,%
    bookmarksnumbered=true,%
    bookmarksopen=true,%
    colorlinks=true,%
    citecolor=blue,%
    linkcolor=blue,%
    anchorcolor=green,%
    urlcolor=blue%
]{hyperref}

\ctexset {
	section = {
		name = {理论},
		format = \bf \zihao{4} \centering
	},
	subsection = {
		name = {},
		number = \Roman{subsection},
		format = \bf \zihao{4} 
	},
	subsubsection = {
		name = {,、\hspace{-0.5em}},
		number = \chinese{subsubsection},
		format = \bf \zihao{4} 
},
paragraph = {
name = {(,)\hspace{-0.5em}},
number = \chinese{paragraph},
format = \bf \zihao{4} 
},
subparagraph = {
name = {,、\hspace{-0.5em}},
format = \bf \zihao{4}
}
}

\setcounter{secnumdepth}{5}


%% 控制目录
\usepackage{titletoc}

%% 支持彩色文本、底色、文本框等
\usepackage{color,xcolor}

%%%% 基本插图方法
%% 图形宏包
\usepackage{graphicx}

%%%% pgf/tikz绘图宏包设置
\usepackage{pgf,tikz}
\usetikzlibrary{shapes,automata,snakes,backgrounds,arrows}
\usetikzlibrary{mindmap}

%%%% fancyhdr设置页眉页脚
%% 页眉页脚宏包
\usepackage{fancyhdr}

\usepackage{amssymb} %常见数学符号

%% 页眉页脚风格
\pagestyle{plain}

%%%% 设置listings宏包用来粘贴源代码
%% 方便粘贴源代码,部分代码高亮功能
\usepackage{listings}

%% 所要粘贴代码的编程语言
%\lstloadlanguages{}

%% 设置listings宏包的一些全局样式
%% 参考http://hi.baidu.com/shawpinlee/blog/item/9ec431cbae28e41cbe09e6e4.html
\lstset{
showstringspaces=false,              %% 设定是否显示代码之间的空格符号
numbers=left,                        %% 在左边显示行号
numberstyle=\tiny,                   %% 设定行号字体的大小
basicstyle=\footnotesize,                    %% 设定字体大小\tiny, \small, \Large等等
keywordstyle=\color{blue!70}, commentstyle=\color{red!50!green!50!blue!50},
                                     %% 关键字高亮
frame=shadowbox,                     %% 给代码加框
rulesepcolor=\color{red!20!green!20!blue!20},
escapechar=`,                        %% 中文逃逸字符,用于中英混排
xleftmargin=2em,xrightmargin=2em, aboveskip=1em,
breaklines,                          %% 这条命令可以让LaTeX自动将长的代码行换行排版
extendedchars=false                  %% 这一条命令可以解决代码跨页时,章节标题,页眉等汉字不显示的问题
}
%%%% listings宏包设置结束

%% 设定段间距
\setlength{\parskip}{0.3\baselineskip}

%% 设定行距
\linespread{1}

\usepackage{tabu} % 用tabu代替 array
\usepackage{multirow}
\usepackage{zhnumber}
\usepackage{calc,marvosym,ifthen,fancybox,url,layout}
\setcounter{tocdepth}{1}
\usepackage{paralist}

%给旁注加个黑原点
\usepackage{wasysym}
\let\marginparNR\marginpar
\def\marginpar#1{\marginparNR{\textcolor{red}{ \CIRCLE{}   #1  }}}

%调整列表前后的间距
\makeatletter
\let\orig@Enumerate=\enumerate
\renewenvironment{enumerate}{\orig@Enumerate}{\vspace{-0.5cm}\endlist}
\let\orig@Itemize=\itemize
\renewenvironment{itemize}{\orig@Itemize}{\vspace{-0.5cm}\endlist}
\makeatother

%给目录进行设定
\titlecontents{section}[0pt]{\addvspace{5pt}\filright}
{ \thecontentslabel \hspace{0.5em} }
{}{\titlerule*[8pt]{.}\contentspage}


%画边框
%\def\boxhack{\leavevmode\vbox to0pt{\vss\rlap{\hskip 320pt
%			\setlength{\unitlength}{1pt}\cornersize*{10pt}\thicklines\fancyoval(365,675)}\vskip -680pt}}
%\def\boxhackb{\leavevmode\vbox to0pt{\vss\rlap{\hskip 80pt
%			\setlength{\unitlength}{1pt}\cornersize*{10pt}\thicklines\fancyoval(100,675)}\vskip -680pt}}

%用tikz画边框	
\def\biankuang{\leavevmode\vbox to0pt{
		\vss\rlap{\hskip 0.8cm
			\tikz \draw(4,0)--(0,0)--(0,-22.5)--(17,-22.5)--(17,0)--(4,0)--(4,-22.5);		
		}\vskip -22.7cm}}

\newcolumntype{M}[1]{>{\zihao{4}\centering\arraybackslash}m{#1}}
\newcolumntype{N}{@{}m{0pt}@{}}

\newcommand{\ktmq}[1]{\gdef\ktmqNR{#1}}%课题名称
\newcommand{\jxmb}[1]{\gdef\jxmbNR{#1}}%教学目标
\newcommand{\jxnd}[1]{\gdef\jxndNR{#1}}%教学难点
\newcommand{\jxzd}[1]{\gdef\jxzdNR{#1}}%教学重点
\newcommand{\jjff}[1]{\gdef\jjffNR{#1}}%解决方法
\newcommand{\jxhj}[1]{\gdef\jxhjNR{#1}}%教学后记

\newcommand{\jc}[1]{\gdef\jcNR{#1}}%教材
\newcommand{\cks}[1]{\gdef\cksNR{#1}}%参考书
\newcommand{\jsxm}[1]{\gdef\jsxmNR{#1}}%教师姓名
\newcommand{\jyszr}[1]{\gdef\jyszrNR{#1}}%教研室主任

\newcommand{\skbc}[1]{\gdef\skbcNR{#1}}%授课班次
\newcommand{\skrq}[1]{\gdef\skrqNR{#1}}%授课日期
\newcommand{\biaoti}[1]{\gdef\biaotiNR{#1}}%标题头

\newcounter{thesectionSY}

\newcommand{\makeshouye}{
	\setcounter{thesectionSY}{\thesection+1}
	\restoregeometry	
	\renewcommand{\headrulewidth}{0pt}
	\pagestyle{fancy}
	\fancyhead{}
	\lhead{} 
	\chead{
		\begin{tabular}{@{\hspace{1.2cm}}M{7cm}@{\hspace{-0.4cm}}M{8cm}N}
			\parbox{7cm}{\linespread{0.2}
				\makebox[7cm][s]{\kaishu \zihao{3} 湖南九嶷职业技术学院}\\ 
				\makebox[7cm][s]{\kaishu \zihao{3} 湖南潇湘技师学院}
			}
			&  \makebox[6cm][s]{\rule{0pt}{0.9cm}\zihao{1} \heiti \kaishu 授课课时计划}\\
		\end{tabular}
	}
	
	\begin{tabular}{M{2.2cm}|M{7cm}|M{5.8cm}N}
		\hline
		\multirow{2}*{
			\rule{0pt}{1.4cm}\parbox[b]{2.cm}{
				\centering 课\hfill 程\hfill 章\hfill 节\\及\hfill 主\hfill 题}}& \heiti \biaotiNR\thethesectionSY  &  ~授~课~教~师\hfill {\heiti \zihao{4} \underline{\jsxmNR}}\hfill 签字~~~&\\[0.6cm] \cline{2-3}
		
		& \heiti \ktmqNR &  ~教研室主任\hfill {\fangsong \bf \zihao{4} \underline{\jyszrNR}}\hfill 签字~~~&\\[0.6cm]\hline
		
		\multicolumn{3}{l}{
			\begin{minipage}[t][3.7cm][t]{15cm}	
				\begin{minipage}[t]{2.5cm}
					\vspace{6pt} \hfill \zihao{4} 教学目标:
				\end{minipage}\hspace{0.5cm}				
				\begin{minipage}[t][4.3cm][t]{12cm}
					\vspace{0pt}\zihao{4} \setlength{\baselineskip}{12pt} 
					\begin{enumerate}[1、]
						\jxmbNR
					\end{enumerate} 
				\end{minipage} 
			\end{minipage}
		}\vspace{0.3cm} &\\ \hline
		\multicolumn{3}{l}{
			\begin{minipage}[t][4cm][t]{15cm}
				\begin{minipage}[t]{2.5cm}
					\vspace{5pt} \hfill \zihao{4} 教学重点:
				\end{minipage}\hspace{0.5cm}				
				\begin{minipage}[t]{12cm}
					\vspace{0pt} \zihao{4} \setlength{\baselineskip}{12pt} 
					\begin{enumerate}[1、] \jxzdNR \end{enumerate}
					\vspace{7pt} 
				\end{minipage}
				\vspace{5pt} 
				\begin{minipage}[t]{2.5cm}
					\vspace{6pt} \hfill \zihao{4} 教学难点:
				\end{minipage}\hspace{0.5cm}		
				\begin{minipage}[t]{12cm}
					\vspace{0pt} \zihao{4} \setlength{\baselineskip}{12pt} 
					\begin{enumerate}[1、] \jxndNR \end{enumerate}
					\vspace{0pt} 	
				\end{minipage}
				\begin{minipage}[t]{2.5cm}
					\vspace{6pt} \hfill \zihao{4} 解决方法:
				\end{minipage}\hspace{0.5cm}		
				\begin{minipage}[t]{12cm}
					\vspace{6pt}\zihao{4} \jjffNR
				\end{minipage}
				
			\end{minipage}
		} &\\  \hline
		
		\multirow{2}*{ 	\rule{0pt}{1.4cm}\parbox[b]{2.cm}{
				\centering 教\hfill 材\hfill 和\\参\hfill 考\hfill 书 } } & \multicolumn{2}{c}{\zihao{4} \jcNR } &\\[0.6cm] \cline{2-3}
		&  \multicolumn{2}{c}{\zihao{4} \cksNR } &\\[0.6cm] \hline
		\multirow{2}*{\rule{0pt}{1.4cm}\parbox[b]{2.cm}{
				\centering 授\hfill 课\hfill 班\hfill 次\\授\hfill 课\hfill 日\hfill 期 } } & \multicolumn{2}{c}{ \skbcNR } &\\[0.6cm] \cline{2-3}
		&\multicolumn{2}{c}{ \skrqNR } &\\[0.6cm] \hline
		
		\multicolumn{3}{l}{
			\begin{minipage}[t]{2.5cm}
				\vspace{0pt} \hfill \zihao{4} 教学后记:
			\end{minipage}\hspace{0.5cm}				
			\begin{minipage}[t][4.5cm][t]{12cm}
				\vspace{0pt}\zihao{4} \jxhjNR
			\end{minipage} 
		}\vspace{0.3cm} &\\ \hline	
	\end{tabular}
	\newpage
	\newgeometry{textwidth={\textwidth-150pt},top=2cm,bottom=2cm,right=2.5cm,includehead,includefoot,marginparsep=28pt,marginparwidth=85pt}
	
	\reversemarginpar
	\fancyhead{} 
	\chead{\hspace{1cm} \kaishu \zihao{1} 教 \hspace{1cm} 案 \hspace{1cm} 纸 }
	%\lhead{\boxhack \boxhackb } %边框 
	\lhead{ \biankuang}%边框 
	\zihao{4}		

	\section{ \ktmqNR }
}

