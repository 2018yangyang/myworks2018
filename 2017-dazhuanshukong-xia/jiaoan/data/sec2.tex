\jxhj{%教学后记
	}
\skrq{%授课日期
	2018年~~ 9月14日~~3-4节}
\ktmq{%课题名称
	程序的基本结构 }
\jxmb{%教学目标,每行前面要加 \item
	\item 掌握数控程序的组成与结构。;

	\item 掌握数控编程的方法;

	\item 掌握编写数控程序的基本思路;

	\item 了解数控常见指令;}
\jxzd{%教学重点,每行前面要加 \item
	\item 数控程序的组成与结构;

	\item 编写数控程序的基本思路;}
\jxnd{%教学难点,每行前面要加 \item
	\item 数控程序的组成与结构;}
\jjff{%教学方法
	通过讲述、举例、演示法来说明;}

\makeshouye %制作教案首页

%%%%教学内容
\subsection{组织教学}
\begin{enumerate}[1、]
	\item 集中学生注意力;
	\item 清查学生人数;
	\item 维持课堂纪律;
\end{enumerate}
\subsection{复习导入及主要内容}
\begin{enumerate}[1、]
	\item 制造自动化的发展;
	\item 数控技术的基本概念;
	\item 数控机床的组成与工作过程;
	\item 数控机床的种类;
	\item 数控机床的坐标系。
\end{enumerate}
\subsection{教学内容及过程}
\subsubsection{数控编程的坐标系及假设}
\begin{enumerate}[1、]
	\item 数控机床的标准坐标系;
	\subitem A、右手笛卡尔直角坐标系;
	\subitem B、数控机床坐标系的判定方法;
	\subitem C、工件静止刀具移动的假设。 
	\item 机床坐标系/机械坐标系;
	\item 工件坐标系;
	\item 零点偏移;
	\item 局部坐标系;
	\item 极坐标系;
\end{enumerate}
\subsubsection{数控程序的结构}
\begin{enumerate}[1、]
	\item 程序展示;
	\item 程序结构;
	\item 程序;
	\item 程序段;
	\item 地址;
	\item 程序段结束符。
\end{enumerate}
\subsubsection{数控程序的指令}
\begin{enumerate}[1、]
	\item 准备功能;
	\item 辅助功能;
	\item 其他;
	\item 模态指令/非模态指令;
	\item 指令分组。
\end{enumerate}
\subsubsection{数控编程的方式}
\begin{enumerate}[1、]
\item 手工编程
利用一般的计算工具,通过各种数学方法,人工进行刀具轨迹的运算,并进行指令编制。该方式比较简单,容易掌握。适用于中等复杂程度、计算量不大的零件编程,对机床操作人员来讲必须掌握的。

手工编程的步骤:

A、图样分析  包括对零件轮廓形状、有关标注(尺寸公差、形状和位置公差及表面质量要求等)及材料和热处理等项要求进行分析。

B、辅助准备  包括确定机床和夹具、机床坐标系、编程坐标系、对刀方法、对刀点位置及机械间隙值等。

C、工艺处理  其内容包括加工余量与分配、刀具的运动方向与加工路线、切削用量及确定程序编制的允许误差等方面。

D、数学处理  包括尺寸分析与作图、选择处理方法、数值计算及对拟合误差的分析和计算等。

E、填写加工程序单  按照数控系统规定的程序格式和要求,填写零件的加工程序单及加工条件等内容。

F、制备控制介质  数控机床在自动输入加工程序时,必须输入用的控制介质。如穿孔带、磁带或软盘等。

H、程序校验  包括对程序单的填写、控制介质的制备、刀具运动轨迹及等项内容所进行的单项或综合校验工作。

手工编程在目前仍是广泛采用的编程方式,即使在自动编程高速发展的将来,手工编程的重要地位也不可取代。

\item 自动编程  

利用计算机(含外围设备)和相应的前置、后置处理程序对程序对零件源程序进行处理,以得到加工程序单各数控带的一种编程方式。

对于曲线轮廓、三维曲面等复杂型面。一般采用自动编程。

在工作站或个人PC上利用CAD/CAM系统进行零件的设计、分析及加工编程。它适用于各类柔性制造系统(FMS)和计算机集成制造系统(CIMS)。

\end{enumerate}
\subsubsection{编写程序的基本思路}
程序初始化(安全保护)--------辅助准备(换刀,主轴启动,切削液开)--------定位到起刀点--------快速下刀--------工进下刀--------走加工轮廓--------提刀---------快速提刀到安全平面-------程序结束(换刀,主轴停止,切削液关,程序返回等)
\subsection{课堂小结}
\begin{enumerate}[1、]
\item 数控编程的坐标系及假设;

\item 数控程序的结构;
\item 数控程序的指令;
\item 数控编程的方式;
\item 编写程序的基本思路。
\end{enumerate}
\vfill
\subsection{布置作业}
\begin{enumerate}[1、]
	\item 写出数控程序的基本结构。
	\item 数控编程的方式有哪些?
\end{enumerate}
\vfill